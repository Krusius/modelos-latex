% ------------------------------------------------------------%
% 2020-12-05
% 2015-2020 - Emerson Ribeiro de Mello - mello@ifsc.edu.br
% ------------------------------------------------------------%
\documentclass[11pt]{../classes/ifscarticle}
\usepackage{../classes/ifscutils}
\geometry{a4paper,hmargin=2cm,top=3.5cm,bottom=2cm,headheight=3cm,heightrounded}


\fancypagestyle{followingpage}{
	\fancyhead[C]{\pgfuseimage{cabecalho}}
}
\pgfdeclareimage[width=1.29\linewidth]{cabecalho}{../classes/imagens/ifsc-header.png}


\begin{document}

\begin{center}
	{\huge Plano de ensino}\vspace*{.5cm}
\end{center}

% \section{Dados da disciplina}
\noindent\fbox{%
\begin{minipage}{.98\textwidth}%
	\begin{small}
	\begin{tabular}{@{}l l@{}}
	\textbf{Curso}                & Engenharia de Telecomunicações\\
	\textbf{Unidade curricular}   & Programação Orientada a Objetos\\
	\textbf{Semestre}             & 2020-01\\
	\textbf{Carga horária}        & 72 horas\\
	\textbf{Professor}            & Emerson Ribeiro de Mello\\
	\textbf{Página da disciplina} & \url{http://docente.ifsc.edu.br/mello}\\
	\end{tabular}
	\end{small}			
\end{minipage}%
}\vspace*{.5cm}

\section{Ementa}

Introdução ao paradigma da orientação a objetos: Classes, objeto, associações entre classes, herança. Introdução à linguagem de modelagem unificada (UML): Diagramas de caso de uso, classes, sequência. Introdução a linguagem de programação Java: Tipos de dados primitivos, estruturas de controle, vetores; concepção de projeto orientado a objetos, herança, polimorfismo; interfaces gráficas amigáveis.

\section{Objetivos}

Ao término da disciplina o aluno será capaz de modelar, implementar e testar software de média complexidade na linguagem Java e de acordo com o paradigma da programação orientada a objetos. Os objetivos específicos da disciplina são:

\begin{itemize}
    \item Introduzir os conceitos da programação orientada a objetos;
    \item Apresentar a linguagem de programação Java e a linguagem de modelagem unificada (UML);
    \item Usar de forma efetiva ferramentas como ambiente integrado de desenvolvimento e sistema de controle de versão para trabalhar de forma colaborativa;
    \item Modelar software de média complexidade por meio de diagramas UML comportamentais e estruturais.
\end{itemize}



\section{Metodologia}

\section{Conteúdo programático}



\nocite{*}

\bibliographystyle{plain}
\renewcommand{\refname}{Bibliografia}
\bibliography{referencias-poo.bib}


\end{document}