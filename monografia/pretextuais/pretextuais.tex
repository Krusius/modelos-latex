% Folha de rosto - (o * indica que haverá a ficha bibliográfica)
\imprimirfolhaderosto*


%-----------------------------------------------%
% ATENÇÃO - Pergunte para a Biblioteca do IFSC
% Inserir a ficha bibliografica - 
%
%-----------------------------------------------%
% Isto é um exemplo de Ficha Catalográfica, ou ``Dados internacionais de
% catalogação-na-publicação''. Você pode utilizar este modelo como referência. 
% Porém, provavelmente a biblioteca da sua universidade lhe fornecerá um PDF
% com a ficha catalográfica definitiva após a defesa do trabalho. Quando estiver
% com o documento, salve-o como PDF no diretório do seu projeto e substitua todo
% o conteúdo de implementação deste arquivo pelo comando abaixo:
%
% \begin{fichacatalografica}
%     \includepdf{fig_ficha_catalografica.pdf}
% \end{fichacatalografica}

% \begin{fichacatalografica}
% 	\sffamily
% 	\vspace*{\fill}					% Posição vertical
% 	\begin{center}					% Minipage Centralizado
% 	\fbox{\begin{minipage}[c][8cm]{13.5cm}		% Largura
% 	\small
% 	\imprimirautor
% 	%Sobrenome, Nome do autor
	
% 	\hspace{0.5cm} \imprimirtitulo  / \imprimirautor. --
% 	\imprimirlocal, \imprimirdata-
	
% 	\hspace{0.5cm} \pageref{LastPage} p. : il. (algumas color.) ; 30 cm.\\
	
% 	\hspace{0.5cm} \imprimirorientadorRotulo~\imprimirorientador\\
	
% 	\hspace{0.5cm}
% 	\parbox[t]{\textwidth}{\imprimirtipotrabalho~--~\imprimirinstituicao,
% 	\imprimirdata.}\\
	
% 	\hspace{0.5cm}
% 		1. Palavra-chave1.
% 		2. Palavra-chave2.
% 		2. Palavra-chave3.
% 		I. Orientador.
% 		II. Instituto Federal de Santa Catarina.
% 		III. Campus São José.
% 		IV. Título 
% 	\end{minipage}}
% 	\end{center}
% \end{fichacatalografica}
%-----------------------------------------------%

%-----------------------------------------------%
% Inserir folha de aprovação
%-----------------------------------------------%

% Isto é um exemplo de Folha de aprovação, elemento obrigatório da NBR
% 14724/2011 (seção 4.2.1.3). Você pode utilizar este modelo até a aprovação
% do trabalho. Após isso, substitua todo o conteúdo deste arquivo por uma
% imagem da página assinada pela banca com o comando abaixo:
%
% \includepdf{folhadeaprovacao_final.pdf}
%
\begin{folhadeaprovacao}

  \begin{center}
    {\ABNTEXchapterfont\large\MakeUppercase{\imprimirautor}}

    \vspace*{\fill}\vspace*{\fill}
    \begin{center}
      \ABNTEXchapterfont\bfseries\Large\MakeUppercase{\imprimirtitulo}
    \end{center}
    \vspace*{\fill}
    
    \imprimirtextoaprovacao
     
    \vspace*{1cm}
    
	\imprimirlocal, 20 de agosto de 2022:

    \vspace*{\fill}

   \end{center}
        


   \assinatura{\textbf{\imprimirorientador} \\ Orientador\\Instituto Federal de Santa Catarina} 
   \assinatura{\textbf{Professor Fulano, Dr.} \\ Instituto X }
   \assinatura{\textbf{Professora Fulana, Dra. } \\ Instituto Y}
   \assinatura{\textbf{Professor Beltrano, Dr.} \\ Instituto Z}
   %\assinatura{\textbf{Professor} \\ Convidado 4}
      
    \vspace*{1cm}  
  
\end{folhadeaprovacao}
%-----------------------------------------------%

%-----------------------------------------------%
% Dedicatória
%-----------------------------------------------%
\begin{dedicatoria}
   \vspace*{\fill}
   \begin{flushright}
   \noindent
   \textit{ Este trabalho é dedicado às crianças adultas que,\\
   quando pequenas, sonharam em se tornar cientistas.}\vspace*{2cm}
   \end{flushright}
\end{dedicatoria}
% ---

%-----------------------------------------------%
% Agradecimentos
%-----------------------------------------------%
\begin{agradecimentos}
Os agradecimentos principais são direcionados à Gerald Weber, Miguel Frasson,
Leslie H. Watter, Bruno Parente Lima, Flávio de Vasconcellos Corrêa, Otavio Real
Salvador, Renato Machnievscz\footnote{Os nomes dos integrantes do primeiro
projeto abn\TeX\ foram extraídos de
\url{http://codigolivre.org.br/projects/abntex/}} e todos aqueles que
contribuíram para que a produção de trabalhos acadêmicos conforme
as normas ABNT com \LaTeX\ fosse possível.

Agradecimentos especiais são direcionados ao Centro de Pesquisa em Arquitetura
da Informação\footnote{\url{http://www.cpai.unb.br/}} da Universidade de
Brasília (CPAI), ao grupo de usuários
\emph{latex-br}\footnote{\url{http://groups.google.com/group/latex-br}} e aos
novos voluntários do grupo
\emph{\abnTeX}\footnote{\url{http://groups.google.com/group/abntex2} e
\url{http://www.abntex.net.br/}}~que contribuíram e que ainda
contribuirão para a evolução do \abnTeX.

\end{agradecimentos}
% ---

%-----------------------------------------------%
% Epígrafe
%-----------------------------------------------%
\begin{epigrafe}
    \vspace*{\fill}
	\begin{flushright}
		\textit{Sempre que te perguntarem se podes fazer um trabalho,\\
		respondas que sim e te ponhas em seguida a aprender como se faz.\\
		T. Roosevelt}
	\end{flushright}
\end{epigrafe}
% ---



%-----------------------------------------------%
% RESUMO
%-----------------------------------------------%
\setlength{\absparsep}{18pt} % ajusta o espaçamento dos parágrafos do resumo
\begin{resumo}
 Segundo a \citeonline[3.1-3.2]{NBR6028:2003}, o resumo deve ressaltar o
 objetivo, o método, os resultados e as conclusões do documento. A ordem e a extensão
 destes itens dependem do tipo de resumo (informativo ou indicativo) e do
 tratamento que cada item recebe no documento original. O resumo deve ser
 precedido da referência do documento, com exceção do resumo inserido no
 próprio documento. O resumo deve ser escrito como um parágrafo único, sem utilizar referências bibliográficas e evitando ao máximo, o uso de siglas/abreviações. O resumo deve conter até X palavras, sendo composto das seguintes partes (organização lógica): introdução, objetivos, justificativa, metodologia e resultados esperados. Esta é a sequência lógica, não devendo ser utilizados títulos e subtítulos. Não abuse na contextualização, pois o foco deve ser nos objetivos e nos resultados esperados. (\ldots) As palavras-chave devem figurar logo abaixo do
 resumo, antecedidas da expressão Palavras-chave:, separadas entre si por
 ponto e finalizadas também por ponto.

\textbf{Palavras-chave}: Latex. Abntex. Editoração de texto.
\end{resumo}

%-----------------------------------------------%
% ABSTRACT
%-----------------------------------------------%
\begin{resumo}[Abstract]
 \begin{otherlanguage*}{english}
   This is the english abstract.

   \vspace{\onelineskip}
 
   \noindent 
   \textbf{Keywords}: Latex. Abntex. Text editoration.
 \end{otherlanguage*}
\end{resumo}
%-----------------------------------------------%
